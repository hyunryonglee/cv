%%%%%%%%%%%%%%%%%%%%%%%%%%%%%%%%%%%%%%%%%
% Medium Length Professional CV
% LaTeX Template
% Version 2.0 (8/5/13)
%
% This template has been downloaded from:
% http://www.LaTeXTemplates.com
%
% Original author:
% Trey Hunner (http://www.treyhunner.com/)
%
% Important note:
% This template requires the resume.cls file to be in the same directory as the
% .tex file. The resume.cls file provides the resume style used for structuring the
% document.
%
%%%%%%%%%%%%%%%%%%%%%%%%%%%%%%%%%%%%%%%%%

%----------------------------------------------------------------------------------------
%	PACKAGES AND OTHER DOCUMENT CONFIGURATIONS
%----------------------------------------------------------------------------------------

\documentclass{resume} % Use the custom resume.cls style

\usepackage[utf8]{inputenc}
\usepackage[T1]{fontenc}
\usepackage{lmodern}
\usepackage{verbatim}
\usepackage{multicol}

\usepackage[left=0.75in,top=0.6in,right=0.75in,bottom=0.6in]{geometry} % Document margins

\name{Hyun Ryong (Ryan) Lee} % Your name
%\address{32 Vassar St, Cambridge, MA 02139} % Your address
\address{(607) 793-6859 \\ hrlee@csail.mit.edu} % Your phone number and email

\begin{document}

%----------------------------------------------------------------------------------------
%	RESEARCH INTERESTS
%----------------------------------------------------------------------------------------

%\begin{rSection}{Research Interests}
%
%computer architecture, designing hardware/software for sparse workloads, benchmarking, performance engineering
%
%\end{rSection}

%----------------------------------------------------------------------------------------
%	EDUCATION SECTION
%----------------------------------------------------------------------------------------

\begin{rSection}{Education}

{\bf Massachusetts Institute of Technology, Cambridge, MA} \hfill { September 2016 - Present} \\ 
S.M. in Electrical Engineering and Computer Science \hfill {May 2022} \\
Ph.D. in Electrical Engineering and Computer Science \hfill { \it{Expected August 2025}} \\
Cumulative GPA: 4.8/5

{\bf Cornell University, Ithaca, NY} \hfill { August 2010 - May 2016} \\ 
B.S. in Electrical and Computer Engineering \\
Dean's List, all semesters \\
Cumulative GPA: 4.22/4.3 

\end{rSection}

%----------------------------------------------------------------------------------------
%	EMPLOYMENT SECTION
%----------------------------------------------------------------------------------------

\begin{rSection}{Research and Industry Employment}

\begin{rSubsection}{MIT Computer Science and Artificial Intelligence Lab}{September 2016 - Present}{Research Assistant}{Cambridge, Massachusetts}

\item Supervisor: Daniel Sanchez
\item Developed {\it Gist:} an architectural support for DRAM that reduces main memory accesses for sparse workloads.
    \vspace{-0.2cm}
    \begin{itemize}
    \item Gist accelerates sparse workloads by placing a small programmable engine in each DRAM chip that can autonomously traverses 
        a wide range of sparse data structures, transferring only the necessary data to the core for computation.
    \item Gist speeds up a wide range of graph analytics workloads by gmean 1.6$\times$ over prior state-of-the-art accelerators for sparse workloads,
        with $\leq$1\,\% overhead in DRAM chip area.
    \item Developed a PIN-based cycle-accurate simulator of Gist based on Zsim and DRAMsim3.
    \end{itemize}
    \vspace{0.2cm}
\item Developed {\it Terminus}, a programmable accelerator for operations on sparse data structures.
    \vspace{-0.2cm}
    \begin{itemize}
    \item Terminus accelerates the irregular control-flow and long-latency memory accesses of read and update operations
        on sparse data structures by executing these operations in place of the core.
    \item Terminus speeds up a diverse set of data structures such as hash tables and trees by gmean 7.4$\times$ over the CPU baseline,
        with $\leq$0.5\,\% overhead in the core area.
    \item Gist provides flexible acceleration of graph analytics workloads (gmean 1.6$\times$ speedup over prior state-of-the-art accelerators)
    with $\leq$1\,\% overhead in DRAM chip area.
    \item Developed a PIN-based cycle-accurate simulator of Terminus based on Zsim.
    \end{itemize}
    \vspace{0.2cm}
\item Developed {\it Datamime}, a tool to automatically generate representative benchmarks of production workloads.
    \vspace{-0.2cm}
    \begin{itemize}
        \item Datamime produces representative benchmarks by automatically generating a representative \emph{dataset}
            of the target application using a profile-guided approach.
        \item Datamime produces benchmarks with $\leq$3.2\,\% error in IPC compared to the target application.
        \item Developed an application profiler that uses hardware performance counters and Intel CAT.
        \item Developed a C++/Python based dataset generator that uses Bayesian Optimization.
    \end{itemize}
    \vspace{0.2cm}
\item Designed hardware features and algorithms for the {\it Swarm} multicore architecture.
\end{rSubsection}

\begin{rSubsection}{Apple Inc.}{May 2023 - Aug 2023}{Platform Architecture Intern}{Santa Clara, CA}

\item Developed methodologies to estimate the performance impact of thread migration across CPUs.
\end{rSubsection}

\begin{rSubsection}{IBM Research}{May 2018 - Aug 2018}{Graduate Research Intern}{Yorktown Heights, NY}

\item Developed analytical models and simulators for evaluating the effectiveness of using memristive crossbars
to accelerate irregular algorithms via in-memory computing.
\end{rSubsection}

\begin{rSubsection}{Apple Inc.}{May 2015 - Aug 2015}{Firmware Engineering Intern}{Cupertino, CA}

\item Refactored the codebase of Mac related input devices. 
\end{rSubsection}

\begin{rSubsection}{Computer Systems Laboratory, Cornell University}{Jun 2014 - May 2016}{Undergraduate Research Assistant}{Ithaca, NY}

\item Developed a self-optimizing Solid-State Drive I/O scheduler on the Flash Translation Layer using a reinforcement learning agent. 
\end{rSubsection}

%\begin{rSubsection}{Ministry of National Defense}{Sep 2012 - Jun 2014}{Sergeant}{Seoul, South Korea}
%\item Completed two years of mandatory military service for South Korea
%\end{rSubsection}

\end{rSection}

\vspace{1cm}

%----------------------------------------------------------------------------------------
%	SKILLS
%----------------------------------------------------------------------------------------

\begin{rSection}{Skills}

Simulation and analytical modeling, performance modeling, benchmark development \\
C, C++, CUDA, Python, Pin, and Unix tools. \\

\end{rSection}

%----------------------------------------------------------------------------------------
%	PUBLICATIONS
%----------------------------------------------------------------------------------------

\begin{rSection}{Publications}

{\bf H. Lee}, T. Han, D. Sanchez,
"Gist: Enhancing DRAM with Architectural Support to Reduce Main Memory Accesses for Sparse Workloads",
under submission to \textit{Proceedings of the 58th International Symposium on Microarchitecture (MICRO)}, 2025 %Acceptance rate: 74/351 (21\%)

{\bf H. Lee} and D. Sanchez,
"Terminus: A Programmable Accelerator for Read and Update Operations on Sparse Data Structures",
in \textit{Proceedings of the 57th International Symposium on Microarchitecture (MICRO)}, 2024 %Acceptance rate: 74/351 (21\%)

{\bf H. Lee} and D. Sanchez,
"Datamime: Generating Representative Benchmarks by Automatically Synthesizing Datasets",
in \textit{Proceedings of the 55th International Symposium on Microarchitecture (MICRO)}, 2022 %Acceptance rate: 74/351 (21\%)
(\textbf{IEEE Micro's Top Picks 2023 Honorable Mention})

M.C. Jeffrey, V.A. Ying, S. Subramanian, {\bf H. Lee}, J. Emer, and D. Sanchez,
"Harmonizing Speculative and Non-Speculative Execution in Architectures for Ordered Parallelism",
in \textit{Proceedings of the 51st International Symposium on Microarchitecture (MICRO)}, 2018 %Acceptance rate: 74/351 (21\%)

S. Subramanian, M.C. Jeffrey, M. Abeydeera, {\bf H. Lee}, V.A. Ying, J. Emer, and D. Sanchez, 
"Fractal: An Execution Model for Fine-Grain Nested Speculative Parallelism",
In \textit{Proceedings of the 44th International Symposium on Computer Architecture (ISCA)}, 2017 %Acceptance rate: 54/322 (17\%)

\end{rSection}

%----------------------------------------------------------------------------------------
%	WORKSHOPS AND PRESENTATIONS
%----------------------------------------------------------------------------------------

\begin{rSection}{Workshops}

J. Mukundan, J.B. Kuang, A. Jagmohan, M. Franceschini, H. Hunter, {\bf H. Lee}, S. Bisht, and J.F. Martínez, "Improving I/O Scheduling in FLASH-based SSDs through Foresight", In Industry-Academia Partnership Workshop, October 2015.

\end{rSection}

%----------------------------------------------------------------------------------------
%	HONORS AND AWARDS
%----------------------------------------------------------------------------------------

\begin{rSection}{Honors and Awards}

{\bf Kwanjeong Educational Foundation Fellowship} \hfill {2016 - 2022} \\
\textbf{IEEE Micro's Top Picks Honorable Mention} \hfill {2023}
%{\bf John G. Pertsch Prize} for highest GPA in ECE class of 2016 \hfill{2016} \\
%{\bf Best Undergraduate Poster Award}, Industry-Academia Partnership Workshop \hfill {2015} \\
%{\bf Tau Beta Pi} \hfill {2015} \\
%{\bf John G. Pertsch Prize} for second highest GPA in ECE class of 2016 \hfill {2015} \\
%{\bf Eta Kappa Nu} \hfill {2015} 

\end{rSection}

%----------------------------------------------------------------------------------------
%	TEACHING EXPERIENCE
%----------------------------------------------------------------------------------------

\begin{rSection}{Teaching Experience}
Teaching Assistant, 6.823: Computer Architecture, MIT \hfill {Spring 2019, 2021; Fall 2021, 2023}
\item Taught weekly discussion sections. Prepared and graded quizzes and labs related to 
    various topics in computer architecture, ranging from branch prediction and cache coherence
    to multicore architecture design and simulation.

\end{rSection}




%----------------------------------------------------------------------------------------
%	EXAMPLE SECTION
%----------------------------------------------------------------------------------------

%\begin{rSection}{Section Name}

%Section content\ldots

%\end{rSection}

%----------------------------------------------------------------------------------------

\end{document}
